\begin{exercise}

Dar un ejemplo de una función acotada $f$ y de una función creciente $\alpha$ definidas en $\left[a,b\right]$ tales que $|f|\in\mathcal{R}\left(\alpha\right)$ pero para las que $\int_a^b f\ d\alpha$ no exista.

\end{exercise}

\renewcommand*{\proofname}{Solución}
\begin{proof}

Tomemos
\begin{align*}
f&:\left[0,1\right]\rightarrow \mathbb{R}\\
&\quad x\longmapsto f(x)=\begin{cases}
          1 \quad &\text{si}\ x \in \mathbb{Q} \\
          -1 \quad &\text{si}\ x \in \mathbb{I} \\
     \end{cases}
\end{align*}
y
\begin{align*}
\alpha&:\left[0,1\right]\rightarrow \mathbb{R}\\
&\quad x\longmapsto \alpha(x)=x.
\end{align*}

Tenemos que $|f|$ es la función constante en $1$ en $\left[0,1\right]$ y por el Ejercicio \ref{exercise:7.1} se sigue que $|f|\in\mathcal{R}\left(\alpha\right)$ en $\left[0,1\right]$. Ahora bien, para toda $P\in \mathcal{P}\left[0,1\right]$ se tiene
\begin{align*}
  U\left(P,f,\alpha\right)&=\sum_{k=1}^n M_k\left(f\right)\Delta\alpha_k\\
  &=\sum_{k=1}^n \Delta\alpha_k\\
  &=\sum_{k=1}^n \left(\alpha\left(x_k\right)-\alpha\left(x_{k-1}\right)\right)\\
  &=\sum_{k=1}^n \left(x_k-x_{k-1}\right)\\
  &=1,
\end{align*}
y también
\begin{align*}
  L\left(P,f,\alpha\right)&=\sum_{k=1}^n m_k\left(f\right)\Delta\alpha_k\\
  &=\sum_{k=1}^n -\Delta\alpha_k\\
  &=-\sum_{k=1}^n \Delta\alpha_k\\
  &=-1,
\end{align*}
con lo cual $\upint_0^1 f\ d\alpha=inf\left\lbrace U\left(P,f,\alpha\right):P\in\mathcal{P}\left[0,1\right]\right\rbrace=inf\left\lbrace 1\right\rbrace=1$, y $\lowint_0^1 f\ d\alpha=sup\left\lbrace L\left(P,f,\alpha\right):P\in\mathcal{P}\left[0,1\right]\right\rbrace=sup\left\lbrace -1\right\rbrace=-1$. Por tanto $\lowint_0^1 f\ d\alpha \neq \upint_0^1 f\ d\alpha$ y $f\notin\mathcal{R}\left(\alpha\right)$ en $\left[0,1\right]$. Con lo anterior concluimos que $\int_0^1 f\ d\alpha$ no existe.

\end{proof}
\renewcommand*{\proofname}{Prueba}

