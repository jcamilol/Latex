Sea $\text{\textepsilon}>0$ cualquiera. Entonces $\frac{\text{\textepsilon}}{2}>0$. Existen $P_1,P_2\in\mathcal{P}\left[a,b\right]$ tales que
  \begin{equation*}
    U\left(P_1,f,\alpha\right) < \upint_a^b f\ d\alpha + \frac{\textepsilon}{2},\quad y,\quad U\left(P_2,g,\alpha\right) < \upint_a^b g\ d\alpha + \frac{\textepsilon}{2}.
  \end{equation*}
  Entonces $U\left(P_1,f,\alpha\right)+U\left(P_2,g,\alpha\right)<\upint_a^b f\ d\alpha + \upint_a^b g\ d\alpha + \text{\textepsilon}$. Como $P_1\cup P_2 \in \mathcal{P}\left[a,b\right]$ y es más fina que $P_1$ y que $P_2$, entonces $U\left(P_1\cup P_2,f,\alpha\right)+U\left(P_1\cup P_2,g,\alpha\right) \leq U\left(P_1,f,\alpha\right)+U\left(P_2,g,\alpha\right)$.\\
  Ahora bien, notemos que en cada subintervalo $\left[x_{k-1},x_k\right]$ de $P_1\cup P_2$ tenemos
  \begin{align*}
    M_k\left(f+g\right)&\leq sup\left\lbrace f\left(x\right)+g\left(x\right): x\in\left[x_{k-1},x_k\right]\right\rbrace\\
    &\leq sup\left\lbrace f\left(x\right): x\in\left[x_{k-1},x_k\right]\right\rbrace+sup\left\lbrace g\left(x\right): x\in\left[x_{k-1},x_k\right]\right\rbrace\\
    &=M_k\left(f\right)+M_k\left(g\right),
  \end{align*}
por tanto,
\begin{align*}
  U\left(P_1\cup P_2, f+g, \alpha\right)&= \sum_{k=1}^{n} M_k\left(f+g\right)\Delta\alpha_k\\
  &\leq\sum_{k=1}^{n} \left(M_k\left(f\right)+M_k\left(g\right)\right)\Delta\alpha_k\\
  &=\sum_{k=1}^{n} M_k\left(f\right)\Delta\alpha_k+\sum_{k=1}^{n} M_k\left(g\right)\Delta\alpha_k\\
  &=U\left(P_1\cup P_2, f, \alpha\right)+U\left(P_1\cup P_2, g, \alpha\right).
\end{align*}
También, $\upint_a^b\left(f+g\right)\ d\alpha \leq U\left(P_1\cup P_2, f+g, \alpha\right)$. Por consiguiente,
\begin{equation*}
\upint_a^b \left(f+g\right)\ d\alpha \leq \upint_a^b f\ d\alpha + \upint_a^b g\ d\alpha +  \text{\textepsilon}.
\end{equation*}
Como esto se tiene para $\text{\textepsilon}>0$ arbitrario, concluimos
\begin{equation*}
\upint_a^b \left(f+g\right)\ d\alpha \leq \upint_a^b f\ d\alpha + \upint_a^b g\ d\alpha .
\end{equation*}

