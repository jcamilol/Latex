\documentclass[letterpaper]{article} %único necesario para: crear documento, hacer título, espacios, interlineado


\usepackage{vmargin} %necesario para ajustar las márgenes con \setmargins
%\usepackage{euler} %usa la fuente "euler" para las ecuaciones
\usepackage{amsfonts} %para Z estilizada
\usepackage{pifont} %para más estilos de viñetas
\usepackage{amsmath} %necesario para equation*
\usepackage{amsthm} %necesario para proof enviroment
\usepackage{dsfont} %alguna letra caligráfica
\usepackage{tipa} %Epsilon bonito
\usepackage{graphicx} %para manejar imágenes
\usepackage{wrapfig} %para imágenes en modo "wrap"
\usepackage{lipsum} %para usar el texto de relleno
\graphicspath{ {./Images/} } %la dirección de la cual se sacan las imágenes

\newtheorem{theorem}{Teorema}
\newtheorem{lemma}{Lema}
\newtheorem{exercise}{Ejercicio}[section]
\renewcommand*{\proofname}{Prueba} %Muestra "prueba" en lugar de "proof"
\def\upint{\mathchoice %Define la integral superior e inferior
    {\mkern13mu\overline{\vphantom{\intop}\mkern7mu}\mkern-20mu}%
    {\mkern7mu\overline{\vphantom{\intop}\mkern7mu}\mkern-14mu}%
    {\mkern7mu\overline{\vphantom{\intop}\mkern7mu}\mkern-14mu}%
    {\mkern7mu\overline{\vphantom{\intop}\mkern7mu}\mkern-14mu}%
  \int}
\def\lowint{\mkern3mu\underline{\vphantom{\intop}\mkern7mu}\mkern-10mu\int}

%letterpaper: 21.59cm x 27.94cm
\setmargins{1.8 cm} %margen izquierdo
{1.5 cm} %margen Superior
{17.5 cm} %anchura del texto
{23cm} %altura del texto
{0pt} %altura de los encabezados
{1.5cm} %espacio entre el texto y los encabezados
{0cm} %altura del pie de página
{2 cm} %espacio entre el texto y el pie de página


%\pagenumbering{gobble} %Quita la numeración de las páginas
\renewcommand{\baselinestretch}{1.5} %Aumenta el interlineado a n veces el automático
\relpenalty=9999 %Evita que se rompan las ecuaciones en el cambio de renglón
\binoppenalty=9999

\title{
  \vspace{-1.5cm}
  \textsc{
    \Large{Integración y Series\\ Primera entrega de ejercicios}\\ \vspace{0.2cm}
    \large{Agosto 31 de 2023\\ \vspace{1cm} \underline{Juan Camilo Lozano Suárez}}\\
  }
}

\date{}

\begin{document}

  \maketitle
  \vspace{-1cm}

  Los siguientes lemas serán usados en algunas soluciones:

  \begin{lemma}

Sea $f$ una función monótona en $ \left[ a,b \right] $. Entonces $V_f \left( a,b \right) = |f\left(b\right)-f\left(a\right)|$.

\end{lemma}

\begin{proof}

Analizamos dos casos:
\begin{itemize}
  
  \item[\tiny{\ding{110}}] Supongamos $f \nearrow en \left[ a,b \right]$. Para cualquier partición $P \in \mathcal{P} \left[ a,b \right]$ se tiene $\sum_{k=1}^{n} |\Delta f_k|= \sum_{k=1}^{n} |f\left(x_k\right)-f\left( x_{k-1}\right)| = \sum_{k=1}^{n} \left(f\left(x_k \right)-f\left(x_{k-1} \right)\right)=f \left(b \right)-f\left(a \right)=|f\left(b \right)-f\left(a \right)|$.\\

  \item[\tiny{\ding{110}}] Supongamos $f \searrow en \left[ a,b \right]$. Para cualquier partición $P \in \mathcal{P} \left[ a,b \right]$ se tiene $\sum_{k=1}^{n} |\Delta f_k|= \sum_{k=1}^{n} |f\left(x_k\right)-f\left( x_{k-1}\right)| = \sum_{k=1}^{n} \left(f\left(x_{k-1} \right)-f\left(x_{k} \right)\right)=f \left(a \right)-f\left(b \right)=|f\left(b \right)-f\left(a \right)|$.\\
  
\end{itemize}

En cualquier caso, se tiene $V_f \left(a,b\right)=sup\left\lbrace\sum \left(P\right):P\in \mathcal{P} \left[a,b\right]\right\rbrace=sup\left\lbrace|f\left(b\right)-f\left(a\right)|\right\rbrace=|f\left(b\right)-f\left(a\right)|.$ 

\end{proof}

  \begin{lemma}

Sea $f$ una función continua en $\left[a,b\right]$, tal que $f'$ existe y es acotada en $\left(a,b\right)$. Entonces $f\in VA\left[a,b\right]$.

\end{lemma}

\begin{proof}

Existe $A\geq 0$ tal que $|f\left(c\right)| \leq A$ para todo $c \in \left[a,b\right]$. Sea $P= \left\lbrace x_0, x_1, \dots, x_n \right\rbrace \in \mathcal{P}\left[a,b\right]$ cualquiera. Para cada k=1,\dots, n, por el teorema del valor medio para derivadas, existe $c_k\in \left[x_{k-1},x_k\right]$ tal que 

\begin{equation*}
  f'\left(c_k\right)=\frac{f\left(x_k\right)-f\left(x_{k-1}\right)}{x_k-x_{k-1}}=\frac{\Delta f_k}{x_k-x_{k-1}}.
\end{equation*}

Luego, $\Delta f_ = f'\left(c_k\right)\left(x_k-x_{k-1}\right)$. De este modo,

\begin{align*}
  \sum_{k=1}^{n}|\Delta f_{k}|&=\sum_{k=1}^{n}\left(|f'\left(c_k\right)||x_k-x_{k-1}|\right)\\
  &\leq \sum_{k=1}^{n}A|x_k-x_{k-1}|\\
  &=A\sum_{k=1}^{n}\left(x_k-x_{k-1}\right)\\
  &=A\left(b-a\right),
\end{align*}

con lo cual $f$ es de variación acotada en $\left[a,b\right]$.

\end{proof}

  \setcounter{section}{6}
\setcounter{exercise}{2}
\begin{exercise}

  Probar que una función polinómica $f$ es de variación acotada en todo intervalo compacto $\left[a,b\right]$. Describir un método que permita calcular la variación total de $f$ en $\left[a,b\right]$ conociendo los ceros de la derivada $f'$.

\end{exercise}

\begin{proof}

Sea $f$ una función polinómica en $\left[a,b\right]$. Sabemos que $f$ es continua en $\left[a,b\right]$ y que $f'$ (que también es una función polinómica) existe y es acotada en $\left(a,b\right)$. Por tanto $f$ es de variación acotada en $\left[a,b\right]$.\\
Si $f$ es constante en $\left[a,b\right]$, se tiene $V_f\left(a,b\right)=0$. Supongamos $grado\left(f\right)\geq 1$, de modo que $grado\left(f'\right)\geq 0$ y $f'$ no es el polinomio nulo. Si $f'$ no tiene ceros en $\left[a,b\right]$, se sigue que $f'\left(x\right)>0$ para todo $x$ en $\left[a,b\right]$, o $f'\left(x\right)<0$ para todo $x$ en $\left[a,b\right]$ (si $f'\left(c\right)<0$ y $f'\left(d\right)>0$ para $c,d\in \left[a,b\right]$, por el teorema del valor intermedio $f'$ tendría algún cero en $\left[a,b\right]$). En todo caso, $f$ es monótona en $\left[a,b\right]$ y $V_f\left(a,b\right)=|f\left(b\right)-f\left(a\right)|$. Supongamos que $f'$ tiene ceros en $\left[a,b\right]$. Como $f'$ tiene a lo más $grado\left(f\right)\in \mathds{Z}^{+}$ ceros en $\mathds{R}$, podemos enumerarlos y ordenarlos. Así, sean $x_1<x_2<\dots<x_m$ todos los ceros de $f'$ en $\left[a,b\right]$, y llamemos $x_0:=a$ y $x_{m+1}:=b$. Notemos que en cada subintervalo $\left[x_{k-1},x_k\right]$ (con $k\in \left\lbrace 1,\dots, m+1 \right\rbrace$) la función $f$ es monótona y por tanto $V_f\left(x_{k-1},x_k\right)=|f\left(x_k\right)-f\left(x_{k-1}\right)|$. Por la propiedad aditiva de la variación total, se sigue que

\begin{equation*}
  V_f\left(a,b\right)=\sum_{k=1}^{m+1}V_f\left(x_{k-1},x_k\right)=\sum_{k=1}^{m+1}|f\left(x_k\right)-f\left(x_{k-1}\right)|.
\end{equation*}

\end{proof}

  \setcounter{section}{7}
\setcounter{exercise}{0}

\begin{exercise}
  Probar que $\int_{a}^{b}d\alpha\left(x\right)=\alpha\left(b\right)-\alpha\left(a\right)$, directamente a partir de la definición de integral de Riemann-Stieltjes.
\end{exercise}

\begin{proof}

Sea $\text{\textepsilon}>0$ cualquiera. Tomemos $P_{\text{\textepsilon}}=\left\lbrace a,b\right\rbrace\in\mathcal{P}\left[a,b\right]$ y $f:\left[a,b\right]\text{\textrightarrow}\mathds{R}$ la función constante en $1$. Si $P\supseteq P_{\text{\textepsilon}}$ tenemos
\begin{align*}
  \left| S\left(P,f,\alpha\right)-\left(\alpha\left(b\right)-\alpha\left(a\right)\right)\right|&=\left|\left(\sum_{k=1}^{n}f\left(t_k\right)\Delta\alpha_k\right)-\left(\alpha\left(b\right)-\alpha\left(a\right)\right) \right|\\
  &=\left|\left(\sum_{k=1}^{n} \alpha\left(x_k\right)-\alpha\left(x_{k-1}\right)\right)-\left(\alpha\left(b\right)-\alpha\left(a\right)\right)\right|\\
  &=\left|\left(\alpha\left(x_n\right)-\alpha\left(x_0\right)\right)-\left(\alpha\left(b\right)-\alpha\left(a\right)\right)\right|\\
  &=\left|\left(\alpha\left(b\right)-\alpha\left(a\right)\right)-\left(\alpha\left(b\right)-\alpha\left(a\right)\right)\right|\\
  &=0\\
  &<\text{\textepsilon},
\end{align*}
lo que prueba $\int_{a}^{b}d\alpha\left(x\right)=\alpha\left(b\right)-\alpha\left(a\right)$.
  
\end{proof}

  \begin{exercise}
  Si $f\in \mathcal{R}\left(\alpha\right)$ en $\left[a,b\right]$ y si $\int_a^b f\ d\alpha=0$ para cada f monótona en $\left[a,b\right]$, probar que $\alpha$ es constante en $\left[a,b\right]$.
\end{exercise}

\begin{proof}
  Sea $c\in\left(a,b\right)$ cualquiera. Definimos $f:\left[a,b\right]\rightarrow\mathds{R}$ vía
  $f(x)=\begin{cases}
    0 \quad &\text{si}\ x \leq c \\
    1 \quad &\text{si}\ x >c \\
  \end{cases}$
  \quad para todo $x\in\left[a,b\right]$. Claramente $f$ es monótona en $\left[a,b\right]$, así que por hipótesis $f\in\mathcal{R}\left(\alpha\right)$ en $\left[a,b\right]$ y $\int_a^b f\ d\alpha=0$. Como $f\in\mathcal{R}\left(\alpha\right)$ entonces $\alpha\in\mathcal{R}\left(f\right)$ en $\left[a,b\right]$. Veamos que $\int_a^b \alpha\ df = \alpha\left(c\right)$:
  \begin{itemize}
    \item[]Sea $\text{\textepsilon}>0$ cualquiera. Existe $P_{\text{\textepsilon}}\in \mathcal{P}\left[a,b\right] $ tal que para cualquier $P\supseteq P_{\text{\textepsilon}}$ y para cualquier elección $t_k\in\left[x_{k-1},x_k\right]$ se tiene $|S\left(P,\alpha,f\right)-\int_a^b\alpha\ df|<\text{\textepsilon}$. Tomemos $P=P_{\text{\textepsilon}}\cup\left\lbrace c \right\rbrace=\left\lbrace x_0=a,\dots,x_\gamma=c,\dots,x_n=b \right\rbrace\supseteq P_{\text{\textepsilon}}$ con la elección $t_k=x_{k-1}\in\left[x_{k-1},x_k\right]$. Se tiene
    \begin{align*}
      S\left(P,\alpha,f\right)&=\sum_{k=1}^{n}\alpha\left(x_{k-1}\right)\Delta f_k\\
      &=\sum_{k=1}^{n}\alpha\left(x_{k-1}\right)\left(f\left(x_k\right)-f\left(x_{k-1}\right)\right).
    \end{align*}
    Para $1\leq k\leq \gamma$ tenemos $f\left(x_k\right)-f\left(x_{k-1}\right)=0-0=0$; para $k=\gamma +1$ tenemos $f\left(x_{\gamma+1}\right)-f\left(x_{\gamma}\right)=f\left(x_{\gamma+1}\right)-f\left(c\right)=1-0=1$; para $\gamma+2\leq k\leq n$ tenemos $f\left(x_k\right)-f\left(x_{k-1}\right)=1-1=0$. De este modo $S\left(P,\alpha,f\right)=\alpha\left(x_\gamma\right)=\alpha\left(c\right)$. Así, $\left|\alpha\left(c\right)-\int_a^b \alpha\ df\right|= \left|S\left(P,\alpha,f\right)-\int_a^b \alpha\ df\right|<\text{\textepsilon}.$ Como esto se tiene para $\text{\textepsilon}>0$ arbitrario, se sigue que $\left|\alpha\left(c\right)-\int_a^b\alpha\ df\right|=0$ y $\int_a^b\alpha\ df=\alpha\left(c\right)$.
  \end{itemize}
  Ahora, haciendo integración por partes tenemos
  \begin{equation*}
    0+\alpha\left(c\right)=\int_a^b f\ d\alpha + \int_a^b\alpha\ df= f\left(b\right)\alpha\left(b\right)-f\left(a\right)\alpha\left(a\right)=\alpha\left(b\right),
  \end{equation*}
  y $\alpha\left(c\right)=\alpha\left(b\right)$, para $c\in\left(a,b\right)$ cualquiera. Como además la función constante en $1$ es monótona en $\left[a,b\right]$, tenemos $\alpha\left(b\right)-\alpha\left(a\right)=\int_a^bd\alpha=0$, y $\alpha\left(a\right)=\alpha\left(b\right)$, completando la prueba de que $\alpha$ es constante en $\left[a,b\right]$.
  
\end{proof}


  \setcounter{section}{7}
\setcounter{exercise}{10}

\begin{exercise}

Si $\alpha \nearrow$ en $\left[a,b\right]$, probar que se verifica :

\begin{itemize}
  \item[a)]$\upint_a^b f\ d\alpha = \upint_a^c f\ d\alpha + \upint_c^b f\ d\alpha$, $\left(a<c<b\right)$,
  \item[b)]$\upint_a^b \left(f+g\right)\ d\alpha \leq \upint_a^b f\ d\alpha + \upint_a^b g\ d\alpha$,
  \item[c)]$\lowint_a^b \left(f+g\right)\ d\alpha \geq \lowint_a^b f\ d\alpha + \lowint_a^b g\ d\alpha$.
\end{itemize}

\end{exercise}

\begin{proof}

\begin{itemize}
  \item[a)]Sea $P\in\mathcal{P}\left[a,b\right]$ cualquiera. Tomamos $P'=P\cup\left\lbrace c\right\rbrace$. Supongamos $P'=\left\lbrace a=x_0,\dots,x_\gamma=c,\dots,x_n=b\right\rbrace$. Tomemos $P'_1=\left\lbrace a=x_0,\dots,x_\gamma=c\right\rbrace\in\mathcal{P}\left[a,c\right]$ y $P'_2=\left\lbrace c=x_\gamma,\dots,x_n=b\right\rbrace\in\mathcal{P}\left[c,b\right]$. Notemos que

  \begin{align*}
  U\left(P'_1,f,\alpha\right)+U\left(P'_2,f,\alpha\right)&=\sum_{k=1}^{\gamma}M_k\left(f\right)\Delta\alpha_k+\sum_{k=\gamma+1}^{n}M_k\left(f\right)\Delta\alpha_k\\
  &=\sum_{k=1}^{n}M_k\left(f\right)\Delta\alpha_k\\
  &=U\left(P',f,\alpha\right).
  \end{align*}

Como $P'\supseteq P$ tenemos $U\left(P',f,\alpha\right)\leq U\left(P,f,\alpha\right)$, y por tanto $U\left(P'_1,f,\alpha\right)+U\left(P'_2,f,\alpha\right)\leq U\left(P,f,\alpha\right)$. Ya que $P'_1\in\mathcal{P}\left[a,c\right]$ y $P'_2\in\mathcal{P}\left[c,b\right]$, se sigue

\begin{equation*}
  \upint_a^c f\ d\alpha \leq U\left(P'_1,f,\alpha\right),\quad y,\quad \upint_c^b f\ d\alpha \leq U\left(P'_2,f,\alpha\right).
\end{equation*}

Por tanto

\begin{equation*}
  \upint_a^c f\ d\alpha + \upint_c^b f\ d\alpha \leq U\left(P'_1,f,\alpha\right)+U\left(P'_2,f,\alpha\right)\leq U\left(P,f,\alpha\right).
\end{equation*}

Como lo anterior se tiene para $P\in\mathcal{P}\left[a,b\right]$ arbitraria, $\upint_a^c f\ d\alpha + \upint_c^b f\ d\alpha$ es cota inferior del conjunto $\left\lbrace U\left(P,f,\alpha\right):P\in\mathcal{P}\right\rbrace$, y por tanto

\setcounter{equation}{0}
\begin{equation}
  \upint_a^c f\ d\alpha + \upint_c^b f\ d\alpha \leq \upint_a^b f\ d\alpha.
\end{equation}

Ahora, sea $\text{\textepsilon}>0$ cuaquiera. Entonces $\frac{\text{\textepsilon}}{2}>0$ y existen $P_1\in\mathcal{P}\left[a,c\right]$ y $P_2\in\mathcal{P}\left[c,b\right]$ tales que

\begin{equation*}
  U\left(P_1,f,\alpha\right)<\upint_a^c f\ d\alpha+\frac{\text{\textepsilon}}{2},\quad y,\quad U\left(P_2,f,\alpha\right)<\upint_c^b f\ d\alpha+\frac{\text{\textepsilon}}{2},
\end{equation*}

luego,

\begin{equation*}
  U\left(P_1\cup P_2,f,\alpha\right) = U\left(P_1,f,\alpha\right)+U\left(P_2,f,\alpha\right) < \upint_a^c f\ d\alpha + \upint_c^b f\ d\alpha + \text{\textepsilon}.
\end{equation*}

Como $P_1\cup P_2\in\mathcal{P}\left[a,b\right]$, tenemos $\upint_a^b f\ d\alpha \leq U\left(P_1\cup P_2,f,\alpha\right)$, y

\begin{equation*}
  \upint_a^b f\ d\alpha \leq \upint_a^c f\ d\alpha + \upint_c^b f\ d\alpha +\text{\textepsilon}.
\end{equation*}

Lo anterior vale para $\text{\textepsilon}>0$ arbitrario, por lo que obtenemos

\setcounter{equation}{1}
\begin{equation}
  \upint_a^b f\ d\alpha \leq \upint_a^c f\ d\alpha + \upint_c^b f\ d\alpha.
\end{equation}

De (1) y (2) se concluye $\upint_a^b f\ d\alpha = \upint_a^c f\ d\alpha + \upint_c^b f\ d\alpha.$

\end{itemize}

\end{proof}


  


\end{document}
