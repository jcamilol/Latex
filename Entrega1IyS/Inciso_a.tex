Sea $P\in\mathcal{P}\left[a,b\right]$ cualquiera. Tomamos $P'=P\cup\left\lbrace c\right\rbrace$. Supongamos $P'=\left\lbrace a=x_0,\dots,x_\gamma=c,\dots,x_n=b\right\rbrace$. Tomemos $P'_1=\left\lbrace a=x_0,\dots,x_\gamma=c\right\rbrace\in\mathcal{P}\left[a,c\right]$ y $P'_2=\left\lbrace c=x_\gamma,\dots,x_n=b\right\rbrace\in\mathcal{P}\left[c,b\right]$. Notemos que
  \begin{align*}
  U\left(P'_1,f,\alpha\right)+U\left(P'_2,f,\alpha\right)&=\sum_{k=1}^{\gamma}M_k\left(f\right)\Delta\alpha_k+\sum_{k=\gamma+1}^{n}M_k\left(f\right)\Delta\alpha_k\\
  &=\sum_{k=1}^{n}M_k\left(f\right)\Delta\alpha_k\\
  &=U\left(P',f,\alpha\right).
  \end{align*}
  Como $P'\supseteq P$ tenemos $U\left(P',f,\alpha\right)\leq U\left(P,f,\alpha\right)$, y por tanto $U\left(P'_1,f,\alpha\right)+U\left(P'_2,f,\alpha\right)\leq U\left(P,f,\alpha\right)$. Ya que $P'_1\in\mathcal{P}\left[a,c\right]$ y $P'_2\in\mathcal{P}\left[c,b\right]$, se sigue
  \begin{equation*}
    \upint_a^c f\ d\alpha \leq U\left(P'_1,f,\alpha\right),\quad y,\quad \upint_c^b f\ d\alpha \leq U\left(P'_2,f,\alpha\right).
  \end{equation*}
  Por tanto
  \begin{equation*}
    \upint_a^c f\ d\alpha + \upint_c^b f\ d\alpha \leq U\left(P'_1,f,\alpha\right)+U\left(P'_2,f,\alpha\right)\leq U\left(P,f,\alpha\right).
  \end{equation*}
  Como lo anterior se tiene para $P\in\mathcal{P}\left[a,b\right]$ arbitraria, $\upint_a^c f\ d\alpha + \upint_c^b f\ d\alpha$ es cota inferior del conjunto $\left\lbrace U\left(P,f,\alpha\right):P\in\mathcal{P}\right\rbrace$, y por tanto
  \setcounter{equation}{0}
  \begin{equation}
    \upint_a^c f\ d\alpha + \upint_c^b f\ d\alpha \leq \upint_a^b f\ d\alpha.
  \end{equation}
  Ahora, sea $\text{\textepsilon}>0$ cuaquiera. Entonces $\frac{\text{\textepsilon}}{2}>0$ y existen $P_1\in\mathcal{P}\left[a,c\right]$ y $P_2\in\mathcal{P}\left[c,b\right]$ tales que
  \begin{equation*}
    U\left(P_1,f,\alpha\right)<\upint_a^c f\ d\alpha+\frac{\text{\textepsilon}}{2},\quad y,\quad U\left(P_2,f,\alpha\right)<\upint_c^b f\ d\alpha+\frac{\text{\textepsilon}}{2},
  \end{equation*}
  luego,
  \begin{equation*}
    U\left(P_1\cup P_2,f,\alpha\right) = U\left(P_1,f,\alpha\right)+U\left(P_2,f,\alpha\right) < \upint_a^c f\ d\alpha + \upint_c^b f\ d\alpha + \text{\textepsilon}.
  \end{equation*}
  Como $P_1\cup P_2\in\mathcal{P}\left[a,b\right]$, tenemos $\upint_a^b f\ d\alpha \leq U\left(P_1\cup P_2,f,\alpha\right)$, y
  \begin{equation*}
    \upint_a^b f\ d\alpha \leq \upint_a^c f\ d\alpha + \upint_c^b f\ d\alpha +\text{\textepsilon}.
  \end{equation*}
  Lo anterior vale para $\text{\textepsilon}>0$ arbitrario, por lo que obtenemos
  \setcounter{equation}{1}
  \begin{equation}
    \upint_a^b f\ d\alpha \leq \upint_a^c f\ d\alpha + \upint_c^b f\ d\alpha.
  \end{equation}
  De (1) y (2) se concluye $\upint_a^b f\ d\alpha = \upint_a^c f\ d\alpha + \upint_c^b f\ d\alpha.$
