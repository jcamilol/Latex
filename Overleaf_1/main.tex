\documentclass[letterpaper]{article} %único necesario para: crear documento, hacer título, espacios, interlineado


\usepackage{vmargin} %necesario para ajustar las márgenes con \setmargins
%\usepackage{euler} %usa la fuente "euler" para las ecuaciones
\usepackage{amsfonts} %para Z estilizada
\usepackage{pifont} %para más estilos de viñetas
\usepackage{amsmath} %necesario para equation*
\usepackage{amsthm} %necesario para proof enviroment
\usepackage{dsfont} %alguna letra caligráfica
\usepackage{tipa} %Epsilon bonito
\usepackage{graphicx} %para manejar imágenes
\usepackage{wrapfig} %para imágenes en modo "wrap"
\usepackage{lipsum} %para usar el texto de relleno
\graphicspath{ {./Images/} } %la dirección de la cual se sacan las imágenes

\newtheorem{theorem}{Teorema}[section]
\newtheorem{lemma}[theorem]{Lema}
\newtheorem{exercise}[theorem]{Ejercicio}
\renewcommand*{\proofname}{Prueba} %Muestra "prueba" en lugar de "proof"
\newtheorem{remark}[theorem]{Observación}
\newtheorem{example}[theorem]{Ejemplo}
\newtheorem{definition}[theorem]{Definición}
\def\upint{\mathchoice %Define la integral superior e inferior
    {\mkern13mu\overline{\vphantom{\intop}\mkern7mu}\mkern-20mu}%
    {\mkern7mu\overline{\vphantom{\intop}\mkern7mu}\mkern-14mu}%
    {\mkern7mu\overline{\vphantom{\intop}\mkern7mu}\mkern-14mu}%
    {\mkern7mu\overline{\vphantom{\intop}\mkern7mu}\mkern-14mu}%
  \int}
\def\lowint{\mkern3mu\underline{\vphantom{\intop}\mkern7mu}\mkern-10mu\int}

%letterpaper: 21.59cm x 27.94cm
\setmargins{1.8 cm} %margen izquierdo
{1.5 cm} %margen Superior
{17.5 cm} %anchura del texto
{23cm} %altura del texto
{0pt} %altura de los encabezados
{1.5cm} %espacio entre el texto y los encabezados
{0cm} %altura del pie de página
{2 cm} %espacio entre el texto y el pie de página


%\pagenumbering{gobble} %Quita la numeración de las páginas
\renewcommand{\baselinestretch}{1.5} %Aumenta el interlineado a n veces el automático
\relpenalty=9999 %Evita que se rompan las ecuaciones en el cambio de renglón
\binoppenalty=9999

\title{
  \vspace{-1.5cm}
  \textsc{
    \Large{Integración y Series\\ Notas de Clase}\\ \vspace{0.2cm}
    \large{Septiembre 31 de 2023\\ \vspace{1cm} \underline{Juan Camilo Lozano Suárez}}\\
  }
}

\date{}

\begin{document}
  \maketitle
  \section{El criterio de la integral}
    \begin{theorem}\textbf{(Criterio de la integral)}
  Sea $f$ una función positiva y decreciente definida en el intervalo $[1,\infty)$ tal que $\lim_{x\to\infty}f(x)=0$. Para cualquier $n\in\mathbb{Z}^+$ definimos
  \begin{equation*}
    S_n=\sum_{k=1}^{n}f(k),\hspace{1cm} t_n=\int_{1}^{n}f(x)dx,\hspace{1cm} d_n=s_n-t_n.
  \end{equation*}
  Entonces tenemos:
  \begin{itemize}
    \item[\textbf{i)}] $0<f(n+1)\leq d_{n+1}\leq d_n \leq f(1)$, para cualquier $n\in\mathbb{Z}^+$.
    \item[\textbf{ii)}] $\lim_{n\to\infty}d_n$ existe.
    \item[\textbf{iii)}] $\sum_{n=1}^\infty f(n)$ converge, si y sólo si la sucesión $\lbrace{t_n\rbrace}$ converge.
    \item[\textbf{iv)}] $0\leq d_k-\lim_{n\to\infty} d_n\leq f(k)$, para cualquier $k\in\mathbb{Z}^+$.
  \end{itemize}
\end{theorem}
\begin{proof}
  \begin{itemize}
    \item[\textbf{i)}] Ya que $f$ es positiva en $[1,\infty)$, inmediatamente se tiene $0<f(n+1)$ para todo $n\in\mathbb{Z}^+$. Como $f \searrow$ en $[1,\infty)$, para cada $k\in\mathbb{Z}^+$ tenemos $f(x)\leq f(k)$ para todo $x\in[k,k+1]$, de modo que $\int_k^{k+1}f(x)dx\leq \int_k^{k+1}f(k)dx$. Así, para cada $n\in\mathbb{Z}^+$ obtenemos:
    \begin{align*}
      t_{n+1}&=\int_1^{n+1}f(x)dx\\
      &=\sum_{k=1}^n \int_k^{k+1}f(x)dx\\
      &\leq\sum_{k=1}^n \int_k^{k+1}f(k)dx\\
      &=\sum_{k=1}^n f(k)\int_k^{k+1}dx\\
      &=\sum_{k=1}^n f(k)\\
      &=S_n.
    \end{align*}
    Así, se sigue $-S_n\leq-t_{n+1}$ y $S_{n+1}-S_n\leq S_{n+1}-t_{n+1}=d_{n+1}$, pero $S_{n+1}-S_n=\sum_k^{n+1}f(k)-\sum_{k=1}^nf(k)=f(n+1)$, luego $f(n+1)\leq d_{n+1}$. Por otra parte, para cada $n\in\mathbb{Z}^+$ se tiene $f(x)\geq f(n+1)$ para todo $x\in[n,n+1]$ (nuevamente, porque $f\searrow en [1,\infty)$), por tanto
    \begin{equation*}
      \int_n^{n+1}f(x)dx\geq\int_n^{n+1}f(n+1)dx=f(n+1)\int_n^{n+1}dx=f(n+1),
    \end{equation*}
    y $\int_n^{n+1}f(x)dx-f(n+1)\geq 0$. Así, se obtiene
    \begin{align*}
      d_n-d_{n+1}&=(S_n-t_n)-(S_{n+1}-t_{n+1})\\
      &=(t_{n+1}-t_n)-(S_{n+1}-S_n)\\
      &=\left(\int_1^{n+1}f(x)dx-\int_1^{n}f(x)dx\right)-\left(\sum_{k=1}^{n+1}f(k)-\sum_{k=1}^n f(k)\right)\\
      &=\int_n^{n+1}f(x)dx-f(n+1)\geq 0,
    \end{align*}
    con lo cual $d_{n+1}\leq d_n$. Como lo anterior vale para cualquier $n\in\mathbb{Z}^+$, hemos probado que $\lbrace{d_n\rbrace}$ es una sucesión decreciente, y por tanto para cualquier $n\in\mathbb{Z}^+$ se tiene
    \begin{equation*}
      d_n\leq d_1=S_1-t_1=\sum_{k=1}^1f(k)-\int_1^{1}f(x)dx=f(1),
    \end{equation*}
    lo cual completa la prueba de \textit{\textbf{i)}}.
    \item[\textbf{ii)}] De \textit{\textbf{i)}} se tiene que $\lbrace{d_n\rbrace}$ es una sucesión decreciente y acotada inferiormente por $0$, y por lo tanto $\lbrace{d_n\rbrace}$ converge, es decir, $\lim_{n\to\infty}d_n$ existe.
    \item[\textbf{iii)}] Se tiene que la serie $\sum_{n=1}^\infty f(n)$ converge, si y sólo si su sucesión de sumas parciales $\lbrace{S_n\rbrace}$ converge. Como $\lim_{n\to\infty} d_n=\lim_{n\to\infty}(S_n-t_n)$ existe, si $\lim_{n\to\infty}S_n$ existe, también lo hace $\lim_{n\to\infty}(S_n-(S_n-t_n))=\lim_{n\to\infty}t_n$, y recíprocamente, si $\lim_{n\to\infty}t_n$ existe, también lo hace $\lim_{n\to\infty}((S_n-t_n)+t_n)=\lim_{n\to\infty}S_n$. Así, $\lbrace{t_n\rbrace}$ converge, si y sólo si $\lbrace{S_n\rbrace}$ converge, es decir, si y sólo si $\sum_{n=1}^\infty f(n)$ converge.
    \item[\textbf{iv)}] Sea $n\in\mathbb{Z}^+$ cualquiera. En la prueba de \textit{\textbf{i)}} se dedujo $d_n-d_{n+1}=\int_n^{n+1}f(x)dx-f(n+1)$. Como además tenemos
    \begin{equation*}
      \int_n^{n+1}f(x)dx\leq\int_n^{n+1}f(n)dx=f(n)\int_n^{n+1}dx=f(n),
    \end{equation*}
    tenemos
    \begin{equation*}
      0\leq d_n-d_{n+1}=\int_n^{n+1}f(x)dx-f(n+1)\leq f(n)-f(n+1).
    \end{equation*}
    Como esto vale para $n\in\mathbb{Z}^+$ arbitrario, para cualesquiera $k,\omega\in\mathbb{Z}^+$ con $\omega\geq k$, tendremos
    \begin{equation*}
      0\leq\sum_{n=k}^{\omega}(d_n-d_n+1)\leq\sum_{n=k}^\omega(f(n)-f(n+1)),
    \end{equation*}
    y por lo tanto
    \begin{equation*}
      0\leq\sum_{n=k}^\infty(d_n-d_{n+1})\leq\sum_{n=k}^\infty(f(n)-f(n+1)).
    \end{equation*}
    Notemos además que las series $\sum_{n=k}^\infty(d_n-d_{n+1})$ y $\sum_{n=k}^\infty(f(n)-f(n+1))$ son telescópicas, de modo que
    \begin{equation*}
      \sum_{n=k}^\infty(d_n-d_{n+1})=d_k-\lim_{n\to\infty}d_{n+1}=d_k-\lim_{n\to\infty}d_{n},
    \end{equation*}
    y,
    \begin{equation*}
      \sum_{n=k}^\infty(f(n)-f(n+1))=f(k)-\lim_{n\to\infty}f(n+1)=f(k),
    \end{equation*}
    pues por hipótesis $\lim_{x\to\infty}f(x)=0$. Así, obtenemos
    \begin{equation*}
      0\leq d_k-\lim_{n\to\infty}d_n\leq f(k),
    \end{equation*}
    para $k\in\mathbb{Z}^+$ cualquiera.
  \end{itemize}
\end{proof}

    \begin{remark}
  \begin{itemize}
    \item[\tiny{\ding{110}}] Que la sucesión $\lbrace{t_n\rbrace}$ converja quiere decir que $\lim_{n\to\infty}t_n=\lim_{n\to\infty}\int_1^{n}f(x)dx$ exista, es decir, que la integral impropia $\int_1^{\infty}f(x)dx$ converge. Así, \textit{\textbf{iii)}} nos dice que $\sum_{k=1}^{\infty}f(n)$ converge, si y sólo si $\int_1^{\infty}f(x)dx$ converge; en la práctica, esta es la forma de usar el criterio de la integral para estudiar la convergencia de series.
    \item[\tiny{\ding{110}}] Si llamamos $D=\lim_{n\to\infty}d_n$, entonces $\textit{\textbf{i)}}$ implica $0\leq D\leq f(1)$, y de $\textit{\textbf{iv)}}$ se tiene
    \begin{equation}
    \label{eqn:obs_crit_int}
      0\leq\sum_{k=1}^n f(k)-\int_1^{n}f(x)dx-D\leq f(n)
    \end{equation}
    para cualquier $n\in\mathbb{Z}^+$. Esta desigualdad es extremadamente útil para calcular ciertas sumas finitas mediante integrales.
  \end{itemize}
\end{remark}

    \begin{example}
  Sea $s\in\mathbb{R}$ cualquiera, y estudiemos la convergencia de la serie $\sum_{n=1}^\infty \frac{1}{n^s}$. Si $s\leq 0$, se tiene $\lim_{x\to\infty}\frac{1}{n^s}\neq 0$ y por tanto $\sum_{n=1}^\infty \frac{1}{n^s}$ diverge trivialmente. Supongamos $s>0$ con $s\neq 1$, y consideremos $f:[1,\infty)\to\mathbb{R}$ como la función definida mediante $f(x)=\frac{1}{x^s}$ para cualquier $x\in[1,\infty)$. Tenemos que $f$ es positiva decreciente y $\lim_{x\to\infty}f(x)=\lim_{x\to\infty}\frac{1}{x^s}=0$; por ende podemos usar el criterio de la integral. Se tiene
  \begin{equation*}
    \int_1^\infty\frac{1}{x^s}dx=\lim_{\omega\to\infty}\int_1^\omega\frac{1}{x^s}dx=\lim_{\omega\to\infty}\left.\left(\frac{x^{1-s}}{1-s}\right)\right|_1^\omega=\lim_{\omega\to\infty}\left(\frac{1}{1-s}(w^{1-s}-1)\right);
  \end{equation*}
  este límite converge si $1<s$ y diverge si $0<s<1$. Por el criterio de la integral $\sum_{n=1}^\infty \frac{1}{n^s}$ converge si $s>1$ y diverge si $0<s<1$. Si $s=1$ entonces $\sum_{n=1}^\infty \frac{1}{n^s}=\sum_{n=1}^\infty \frac{1}{n}$, y obtenemos la serie armónica, que es divergente.
\end{example}


  \section{La notación $O$ grande y $o$ pequeña}
    \begin{definition}
  Sean $\lbrace{a_n\rbrace}$ y $\lbrace{b_n\rbrace}$ sucesiones reales con $b_n\geq 0$ para todo $n\in\mathbb{Z}^+$. Escribimos $a_n=O(b_n)$ (léase ``$a_n$ es $O$ grande de $b_n$"), si existe $M\in\mathbb{R}^+$ tal que $|a_n|\leq Mb_n$ para todo $n\in\mathbb{Z}^+$. Escribimos $a_n=o(b_n)$ (léase ``$a_n$ es $o$ pequeña de $b_n$") cuando $n\to\infty$, si $\lim_{n\to\infty}\frac{a_n}{b_n}=0$.
\end{definition}
\begin{remark}
  Una ecuación de la forma $a_n=c_n+O(b_n)$ significa que $a_n-c_n=O(b_n)$. Similarmente, $a_n=c_n+o(b_n)$ significa $a_n-c_n=o(b_n)$. La ventaja de esta notación es que nos permite reemplazar ciertas desigualdades por igualdades. Por ejemplo, la desigualdad \ref{eqn:obs_crit_int} implica
  \begin{equation}
  \label{eqn:Ec_con_O}  
    \sum_{k=1}^{n}f(k)=\int_1^{n}f(x)dx+D+O(f(n)).
  \end{equation}
\end{remark}
\begin{example}
  Tomemos $f(x)=\frac{1}{x}$ en el Teorema \ref{criterio_integral}. Tenemos entonces $t_n=\int_{1}^{n}\frac{1}{x}dx=\log n$, y el inciso \textit{\textbf{ii)}} nos garantiza la existencia del límite
  \begin{equation*} 
    \lim_{n\to\infty}\left(\sum_{k=1}^n\frac{1}{k}-\log n\right),
  \end{equation*} 
  un famoso número conocido como \textit{constante de Euler}, o \textit{constante de Euler-Mascheroni} (no confundir con el \textit{número de Euler} ``e"), y se denota usualmente por $C$ (o por $\gamma$). Así, de la Ecuación \ref{eqn:Ec_con_O} obtenemos:
  \begin{equation*}
    \sum_{k=1}^{n}\frac{1}{k}=\log n+C+O\left(\frac{1}{n}\right).
  \end{equation*}
\end{example}
\begin{example}
  Ahora, dado $s\in\mathbb{R}$ cualquiera, con $s\neq 1$, tomemos $f(x)=\frac{1}{x^s}$ en el Teorema \ref{criterio_integral}. En el ejemplo \ref{Ejemplo_pseries} probamos que la serie $\sum_{n=1}^{\infty}\frac{1}{n^s}$ converge únicamente cuando $s>1$. Para $s>1$, dicha serie define una importante función conocida como la \textit{función zeta de Riemann}:
  \begin{equation*}
    \zeta(s)=\sum_{n=1}^{\infty}\frac{1}{n^s}\hspace{1cm}(s>1).
  \end{equation*}
  Para $s>0, s\neq 1$, podemos aplicar la Ecuación \ref{eqn:Ec_con_O} para obtener
  \begin{equation*}
    \sum_{k=1}^{n}\frac{1}{k^s}=\frac{n^{1-s}-1}{1-s}+C(s)+O\left(\frac{1}{n^s}\right),
  \end{equation*}
  donde $C(s)=\lim_{n\to\infty}\left(\left(\sum_{k=1}^{n}\frac{1}{k^s}\right)-\frac{n^{1-s}-1}{1-s}\right)$.
\end{example}

\end{document}
