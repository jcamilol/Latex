\documentclass[letterpaper]{article} %único necesario para: crear documento, hacer título, espacios, interlineado


\usepackage{vmargin} %necesario para ajustar las márgenes con \setmargins
%\usepackage{euler} %usa la fuente "euler" para las ecuaciones
\usepackage{amsfonts} %para Z estilizada
\usepackage{pifont} %para más estilos de viñetas
\usepackage{amsmath} %necesario para equation*
\usepackage{amsthm} %necesario para proof enviroment
\usepackage{dsfont} %alguna letra caligráfica
\usepackage{tipa} %Epsilon bonito
\usepackage{graphicx} %para manejar imágenes
\usepackage{wrapfig} %para imágenes en modo "wrap"
\usepackage{lipsum} %para usar el texto de relleno
\graphicspath{ {./Images/} } %la dirección de la cual se sacan las imágenes

\newtheorem{theorem}{Teorema}[section]
\newtheorem{lemma}[theorem]{Lema}
\newtheorem{exercise}[theorem]{Ejercicio}
\renewcommand*{\proofname}{Prueba} %Muestra "prueba" en lugar de "proof"
\newtheorem{remark}[theorem]{Observación}
\newtheorem{example}[theorem]{Ejemplo}
\newtheorem{definition}[theorem]{Definición}
\def\upint{\mathchoice %Define la integral superior e inferior
    {\mkern13mu\overline{\vphantom{\intop}\mkern7mu}\mkern-20mu}%
    {\mkern7mu\overline{\vphantom{\intop}\mkern7mu}\mkern-14mu}%
    {\mkern7mu\overline{\vphantom{\intop}\mkern7mu}\mkern-14mu}%
    {\mkern7mu\overline{\vphantom{\intop}\mkern7mu}\mkern-14mu}%
  \int}
\def\lowint{\mkern3mu\underline{\vphantom{\intop}\mkern7mu}\mkern-10mu\int}

%letterpaper: 21.59cm x 27.94cm
\setmargins{1.8 cm} %margen izquierdo
{1.5 cm} %margen Superior
{17.5 cm} %anchura del texto
{23cm} %altura del texto
{0pt} %altura de los encabezados
{1.5cm} %espacio entre el texto y los encabezados
{0cm} %altura del pie de página
{2 cm} %espacio entre el texto y el pie de página


%\pagenumbering{gobble} %Quita la numeración de las páginas
\renewcommand{\baselinestretch}{1.5} %Aumenta el interlineado a n veces el automático
\relpenalty=9999 %Evita que se rompan las ecuaciones en el cambio de renglón
\binoppenalty=9999

\title{
  \vspace{-1.5cm}
  \textsc{
    \Large{Integración y Series\\ Notas de Clase}\\ \vspace{0.2cm}
    \large{Septiembre 31 de 2023\\ \vspace{1cm} \underline{Juan Camilo Lozano Suárez}}\\
  }
}

\date{}

\begin{document}
  \maketitle
  \section{El criterio de la integral}
    \begin{theorem}\textbf{(Criterio de la integral)}
  Sea $f$ una función positiva y decreciente definida en el intervalo $[1,\infty)$ tal que $\lim_{x\to\infty}f(x)=0$. Para cualquier $n\in\mathbb{Z}^+$ definimos
  \begin{equation*}
    S_n=\sum_{k=1}^{n}f(k),\hspace{1cm} t_n=\int_{1}^{n}f(x)dx,\hspace{1cm} d_n=s_n-t_n.
  \end{equation*}
  Entonces tenemos:
  \begin{itemize}
    \item[\textbf{i)}] $0<f(n+1)\leq d_{n+1}\leq d_n \leq f(1)$, para cualquier $n\in\mathbb{Z}^+$.
    \item[\textbf{ii)}] $\lim_{n\to\infty}d_n$ existe.
    \item[\textbf{iii)}] $\sum_{n=1}^\infty f(n)$ converge, si y sólo si la sucesión $\lbrace{t_n\rbrace}$ converge.
    \item[\textbf{iv)}] $0\leq d_k-\lim_{n\to\infty} d_n\leq f(k)$, para cualquier $k\in\mathbb{Z}^+$.
  \end{itemize}
\end{theorem}
\begin{proof}
  \begin{itemize}
    \item[\textbf{i)}] Ya que $f$ es positiva en $[1,\infty)$, inmediatamente se tiene $0<f(n+1)$ para todo $n\in\mathbb{Z}^+$. Como $f \searrow$ en $[1,\infty)$, para cada $k\in\mathbb{Z}^+$ tenemos $f(x)\leq f(k)$ para todo $x\in[k,k+1]$, de modo que $\int_k^{k+1}f(x)dx\leq \int_k^{k+1}f(k)dx$. Así, para cada $n\in\mathbb{Z}^+$ obtenemos:
    \begin{align*}
      t_{n+1}&=\int_1^{n+1}f(x)dx\\
      &=\sum_{k=1}^n \int_k^{k+1}f(x)dx\\
      &\leq\sum_{k=1}^n \int_k^{k+1}f(k)dx\\
      &=\sum_{k=1}^n f(k)\int_k^{k+1}dx\\
      &=\sum_{k=1}^n f(k)\\
      &=S_n.
    \end{align*}
    Así, se sigue $-S_n\leq-t_{n+1}$ y $S_{n+1}-S_n\leq S_{n+1}-t_{n+1}=d_{n+1}$, pero $S_{n+1}-S_n=\sum_k^{n+1}f(k)-\sum_{k=1}^nf(k)=f(n+1)$, luego $f(n+1)\leq d_{n+1}$. Por otra parte, para cada $n\in\mathbb{Z}^+$ se tiene $f(x)\geq f(n+1)$ para todo $x\in[n,n+1]$ (nuevamente, porque $f\searrow en [1,\infty)$), por tanto
    \begin{equation*}
      \int_n^{n+1}f(x)dx\geq\int_n^{n+1}f(n+1)dx=f(n+1)\int_n^{n+1}dx=f(n+1),
    \end{equation*}
    y $\int_n^{n+1}f(x)dx-f(n+1)\geq 0$. Así, se obtiene
    \begin{align*}
      d_n-d_{n+1}&=(S_n-t_n)-(S_{n+1}-t_{n+1})\\
      &=(t_{n+1}-t_n)-(S_{n+1}-S_n)\\
      &=\left(\int_1^{n+1}f(x)dx-\int_1^{n}f(x)dx\right)-\left(\sum_{k=1}^{n+1}f(k)-\sum_{k=1}^n f(k)\right)\\
      &=\int_n^{n+1}f(x)dx-f(n+1)\geq 0,
    \end{align*}
    con lo cual $d_{n+1}\leq d_n$. Como lo anterior vale para cualquier $n\in\mathbb{Z}^+$, hemos probado que $\lbrace{d_n\rbrace}$ es una sucesión decreciente, y por tanto para cualquier $n\in\mathbb{Z}^+$ se tiene
    \begin{equation*}
      d_n\leq d_1=S_1-t_1=\sum_{k=1}^1f(k)-\int_1^{1}f(x)dx=f(1),
    \end{equation*}
    lo cual completa la prueba de \textit{\textbf{i)}}.
    \item[\textbf{ii)}] De \textit{\textbf{i)}} se tiene que $\lbrace{d_n\rbrace}$ es una sucesión decreciente y acotada inferiormente por $0$, y por lo tanto $\lbrace{d_n\rbrace}$ converge, es decir, $\lim_{n\to\infty}d_n$ existe.
    \item[\textbf{iii)}] Se tiene que la serie $\sum_{n=1}^\infty f(n)$ converge, si y sólo si su sucesión de sumas parciales $\lbrace{S_n\rbrace}$ converge. Como $\lim_{n\to\infty} d_n=\lim_{n\to\infty}(S_n-t_n)$ existe, si $\lim_{n\to\infty}S_n$ existe, también lo hace $\lim_{n\to\infty}(S_n-(S_n-t_n))=\lim_{n\to\infty}t_n$, y recíprocamente, si $\lim_{n\to\infty}t_n$ existe, también lo hace $\lim_{n\to\infty}((S_n-t_n)+t_n)=\lim_{n\to\infty}S_n$. Así, $\lbrace{t_n\rbrace}$ converge, si y sólo si $\lbrace{S_n\rbrace}$ converge, es decir, si y sólo si $\sum_{n=1}^\infty f(n)$ converge.
    \item[\textbf{iv)}] Sea $n\in\mathbb{Z}^+$ cualquiera. En la prueba de \textit{\textbf{i)}} se dedujo $d_n-d_{n+1}=\int_n^{n+1}f(x)dx-f(n+1)$. Como además tenemos
    \begin{equation*}
      \int_n^{n+1}f(x)dx\leq\int_n^{n+1}f(n)dx=f(n)\int_n^{n+1}dx=f(n),
    \end{equation*}
    tenemos
    \begin{equation*}
      0\leq d_n-d_{n+1}=\int_n^{n+1}f(x)dx-f(n+1)\leq f(n)-f(n+1).
    \end{equation*}
    Como esto vale para $n\in\mathbb{Z}^+$ arbitrario, para cualesquiera $k,\omega\in\mathbb{Z}^+$ con $\omega\geq k$, tendremos
    \begin{equation*}
      0\leq\sum_{n=k}^{\omega}(d_n-d_n+1)\leq\sum_{n=k}^\omega(f(n)-f(n+1)),
    \end{equation*}
    y por lo tanto
    \begin{equation*}
      0\leq\sum_{n=k}^\infty(d_n-d_{n+1})\leq\sum_{n=k}^\infty(f(n)-f(n+1)).
    \end{equation*}
    Notemos además que las series $\sum_{n=k}^\infty(d_n-d_{n+1})$ y $\sum_{n=k}^\infty(f(n)-f(n+1))$ son telescópicas, de modo que
    \begin{equation*}
      \sum_{n=k}^\infty(d_n-d_{n+1})=d_k-\lim_{n\to\infty}d_{n+1}=d_k-\lim_{n\to\infty}d_{n},
    \end{equation*}
    y,
    \begin{equation*}
      \sum_{n=k}^\infty(f(n)-f(n+1))=f(k)-\lim_{n\to\infty}f(n+1)=f(k),
    \end{equation*}
    pues por hipótesis $\lim_{x\to\infty}f(x)=0$. Así, obtenemos
    \begin{equation*}
      0\leq d_k-\lim_{n\to\infty}d_n\leq f(k),
    \end{equation*}
    para $k\in\mathbb{Z}^+$ cualquiera.
  \end{itemize}
\end{proof}

    \begin{remark}
  \begin{itemize}
    \item[\tiny{\ding{110}}] Que la sucesión $\lbrace{t_n\rbrace}$ converja quiere decir que $\lim_{n\to\infty}t_n=\lim_{n\to\infty}\int_1^{n}f(x)dx$ exista, es decir, que la integral impropia $\int_1^{\infty}f(x)dx$ converge. Así, \textit{\textbf{iii)}} nos dice que $\sum_{k=1}^{\infty}f(n)$ converge, si y sólo si $\int_1^{\infty}f(x)dx$ converge; en la práctica, esta es la forma de usar el criterio de la integral para estudiar la convergencia de series.
    \item[\tiny{\ding{110}}] Si llamamos $D=\lim_{n\to\infty}d_n$, entonces $\textit{\textbf{i)}}$ implica $0\leq D\leq f(1)$, y de $\textit{\textbf{iv)}}$ se tiene
    \begin{equation}
    \label{eqn:obs_crit_int}
      0\leq\sum_{k=1}^n f(k)-\int_1^{n}f(x)dx-D\leq f(n)
    \end{equation}
    para cualquier $n\in\mathbb{Z}^+$. Esta desigualdad es extremadamente útil para calcular ciertas sumas finitas mediante integrales.
  \end{itemize}
\end{remark}

    \begin{example}
  Sea $s\in\mathbb{R}$ cualquiera, y estudiemos la convergencia de la serie $\sum_{n=1}^\infty \frac{1}{n^s}$. Si $s\leq 0$, se tiene $\lim_{x\to\infty}\frac{1}{n^s}\neq 0$ y por tanto $\sum_{n=1}^\infty \frac{1}{n^s}$ diverge trivialmente. Supongamos $s>0$ con $s\neq 1$, y consideremos $f:[1,\infty)\to\mathbb{R}$ como la función definida mediante $f(x)=\frac{1}{x^s}$ para cualquier $x\in[1,\infty)$. Tenemos que $f$ es positiva decreciente y $\lim_{x\to\infty}f(x)=\lim_{x\to\infty}\frac{1}{x^s}=0$; por ende podemos usar el criterio de la integral. Se tiene
  \begin{equation*}
    \int_1^\infty\frac{1}{x^s}dx=\lim_{\omega\to\infty}\int_1^\omega\frac{1}{x^s}dx=\lim_{\omega\to\infty}\left.\left(\frac{x^{1-s}}{1-s}\right)\right|_1^\omega=\lim_{\omega\to\infty}\left(\frac{1}{1-s}(w^{1-s}-1)\right);
  \end{equation*}
  este límite converge si $1<s$ y diverge si $0<s<1$. Por el criterio de la integral $\sum_{n=1}^\infty \frac{1}{n^s}$ converge si $s>1$ y diverge si $0<s<1$. Si $s=1$ entonces $\sum_{n=1}^\infty \frac{1}{n^s}=\sum_{n=1}^\infty \frac{1}{n}$, y obtenemos la serie armónica, que es divergente.
\end{example}


  \section{La notación $O$ grande y $o$ pequeña}
    \begin{definition}
  Sean $\lbrace{a_n\rbrace}$ y $\lbrace{b_n\rbrace}$ sucesiones reales con $b_n\geq 0$ para todo $n\in\mathbb{Z}^+$. Escribimos $a_n=O(b_n)$ (léase ``$a_n$ es $O$ grande de $b_n$"), si existe $M\in\mathbb{R}^+$ tal que $|a_n|\leq Mb_n$ para todo $n\in\mathbb{Z}^+$. Escribimos $a_n=o(b_n)$ (léase ``$a_n$ es $o$ pequeña de $b_n$") cuando $n\to\infty$, si $\lim_{n\to\infty}\frac{a_n}{b_n}=0$.
\end{definition}
\begin{remark}
  Una ecuación de la forma $a_n=c_n+O(b_n)$ significa que $a_n-c_n=O(b_n)$. Similarmente, $a_n=c_n+o(b_n)$ significa $a_n-c_n=o(b_n)$. La ventaja de esta notación es que nos permite reemplazar ciertas desigualdades por igualdades. Por ejemplo, la desigualdad \ref{eqn:obs_crit_int} implica
  \begin{equation}
  \label{eqn:Ec_con_O}  
    \sum_{k=1}^{n}f(k)=\int_1^{n}f(x)dx+D+O(f(n)).
  \end{equation}
\end{remark}
\begin{example}
  Tomemos $f(x)=\frac{1}{x}$ en el Teorema \ref{criterio_integral}. Tenemos entonces $t_n=\int_{1}^{n}\frac{1}{x}dx=\log n$, y el inciso \textit{\textbf{ii)}} nos garantiza la existencia del límite
  \begin{equation*} 
    \lim_{n\to\infty}\left(\sum_{k=1}^n\frac{1}{k}-\log n\right),
  \end{equation*} 
  un famoso número conocido como \textit{constante de Euler}, o \textit{constante de Euler-Mascheroni} (no confundir con el \textit{número de Euler} ``e"), y se denota usualmente por $C$ (o por $\gamma$). Así, de la Ecuación \ref{eqn:Ec_con_O} obtenemos:
  \begin{equation*}
    \sum_{k=1}^{n}\frac{1}{k}=\log n+C+O\left(\frac{1}{n}\right).
  \end{equation*}
\end{example}
\begin{example}
  Ahora, dado $s\in\mathbb{R}$ cualquiera, con $s\neq 1$, tomemos $f(x)=\frac{1}{x^s}$ en el Teorema \ref{criterio_integral}. En el ejemplo \ref{Ejemplo_pseries} probamos que la serie $\sum_{n=1}^{\infty}\frac{1}{n^s}$ converge únicamente cuando $s>1$. Para $s>1$, dicha serie define una importante función conocida como la \textit{función zeta de Riemann}:
  \begin{equation*}
    \zeta(s)=\sum_{n=1}^{\infty}\frac{1}{n^s}\hspace{1cm}(s>1).
  \end{equation*}
  Para $s>0, s\neq 1$, podemos aplicar la Ecuación \ref{eqn:Ec_con_O} para obtener
  \begin{equation*}
    \sum_{k=1}^{n}\frac{1}{k^s}=\frac{n^{1-s}-1}{1-s}+C(s)+O\left(\frac{1}{n^s}\right),
  \end{equation*}
  donde $C(s)=\lim_{n\to\infty}\left(\left(\sum_{k=1}^{n}\frac{1}{k^s}\right)-\frac{n^{1-s}-1}{1-s}\right)$.
\end{example}

  \section{Límite superior y límite inferior de una sucesión de números reales}
    En el curso de Introducción al Análisis real se estudió el concepto de límite superior y límite inferior de una sucesión con valores reales. En la presente sección hacemos un pequeño recuerdo de las definiciones y algunos teoremas que se usarán más adelante.

\begin{definition}
  Sea $\lbrace{a_n\rbrace}$ una sucesión de números reales. Suponga que existe $U\in\mathbb{R}$ que satisface las siguientes dos condiciones:
  \begin{itemize}
    \item[\textbf{i)}] Para todo $\text{\textepsilon}>0$ existe $N\in\mathbb{Z}^+$ tal que para cualquier $n>N$ se tiene
    \begin{equation*}
      a_n<U+\text{\textepsilon}.
    \end{equation*}
    \item[\textbf{ii)}]Para todo $\text{\textepsilon}>0$ y para todo entero $m>0$ existe un entero $n>m$ tal que
    \begin{equation*}
      a_n>U-\text{\textepsilon}.
    \end{equation*}
    Entonces $U$ es llamado el límite superior de $\lbrace{a_n\rbrace}$ y escribimos
    \begin{equation*}
      U=\limsup_{n\to\infty}a_n.
    \end{equation*}
  \end{itemize}
  El inciso $\textit{\textbf{(i)}}$ implica que la sucesión $\lbrace{a_n\rbrace}$ está aacotada superiormente; si no lo está, definimos $\limsup_{n\to\infty}a_n=\infty$. Si la sucesión está acotada superiormente pero no inferiormente, y si no tiene límite superior finito, definimos $\limsup_{n\to\infty}=-\infty$. El límite inferior de $\lbrace{a_n\rbrace}$ se define como:
  \begin{equation*}
    \liminf_{n\to\infty}a_n=-\limsup_{n\to\infty}(-a_n).
  \end{equation*}
\end{definition}
\begin{remark}
  El inciso $\textit{\textbf{(i)}}$ significa que a partir de cierto punto todos los términos de la sucesión $\lbrace{a_n\rbrace}$ estarán a la izquierda de $U+\text{\textepsilon}$; esto también lo expresamos diciendo que ``casi toda la sucesión" está a la izquierda de $U+\text{\textepsilon}$. El inciso \textit{\textbf{(ii)}} significa que hay infinitos términos de la sucesión a la derecha de $U-\text{\textepsilon}$.
\end{remark}
\begin{theorem}
  Sea $\left\{a_n\right\}$ una sucesión de números reales. Entonces se tiene:
  \begin{itemize}
    \item[\textbf{a)}] $\liminf _{n \rightarrow \infty} a_n \leq \lim\sup _{n \rightarrow \infty} a_n$.
    \item[\textbf{b)}] La sucesión $\lbrace{a_n\rbrace}$ converge si, y sólo si, $\limsup _{n \rightarrow \infty} a_n$ y $\liminf _{n \rightarrow \infty} a_n$ son ambos finitos e iguales; en este caso, $\lim _{n \rightarrow \infty} a_n=\liminf _{n \rightarrow \infty} a_n=\limsup _{n \rightarrow \infty} a_n$.
    \item[\textbf{c)}] La sucesión $\lbrace{a_n\rbrace}$ diverge hacia $\infty$ si, y sólo si, $\liminf _{n \rightarrow \infty} a_n=\limsup _{n \rightarrow \infty} a_n=$ $\infty$.
    \item[\textbf{d)}] La sucesión $\lbrace{a_n\rbrace}$ diverge hacia $-\infty$ si, y sólo si, $\lim\inf _{n \rightarrow \infty} a_n=\limsup _{n \rightarrow \infty} a_n=$ $-\infty$.
  \end{itemize}
\end{theorem}
\begin{theorem}
  Suponga que $a_n \leq b_n$ para todo $n\in\mathbb{Z}^+$. Entonces se tiene:
  $$
    \liminf _{n \rightarrow \infty} a_n \leq \liminf _{n \rightarrow \infty} b_n \quad \text { y } \quad \limsup _{n \rightarrow \infty} a_n \leq \limsup _{n \rightarrow \infty} b_n .
  $$
\end{theorem}
\begin{example}
  \hspace{6cm} \\
  \begin{tabular}{l  l   l}
    \textbf{1.} $a_n=(-1)^n(1+1 / n)$,&$\liminf _{n \rightarrow \infty} a_n=-1$,&$\limsup a_n=1$.\\
    \textbf{2.} $a_n=(-1)^n$,&$\liminf _{n \rightarrow \infty} a_n=-1$,&$\limsup _{n \rightarrow \infty} a_n=1$.\\
    \textbf{3.} $a_n=(-1)^n n$,&$\liminf _{n \rightarrow \infty} a_n=-\infty$,&$\limsup _{n \rightarrow \infty} a_n=\infty$.\\
    \textbf{4.} $a_n=n^2 \sin ^2\left(\frac{1}{2} n \pi\right)$,&$\liminf _{n \rightarrow \infty} a_n=0$,&$\limsup a_n=\infty$.
  \end{tabular}
\end{example}




  \section{Criterio del cociente y criterio de la raíz}
    \begin{theorem}[Criterio del cociente]
  Sea $\sum a_n$ una serie de números complejos no nulos, y tomemos
  \begin{equation*}
    r=\liminf_{n\to\infty}\left|\frac{a_{n+1}}{a_n}\right|, \hspace{1cm}R=\limsup_{n\to\infty}\left|\frac{a_{n+1}}{a_n}\right|.
  \end{equation*}
  Entonces se tiene que:
\end{theorem}


    

    \begin{example}
  Consideremos la serie 
  $$\sum a_n=\frac{1}{2}+\frac{1}{3}+\frac{1}{2^2}+\frac{1}{3^2}+\frac{1}{2^3}+\frac{1}{3^3}+\frac{1}{2^4}+\frac{1}{3^4}+\dots$$
  Para $n$ par tenemos 
  $$\left|\frac{a_{n+1}}{a_n}\right|=\left|\frac{3^{\frac{n}{2}}}{2^{\frac{n+2}{2}}}\right|=\frac{3^{n / 2}}{2^{n / 2}\cdot 2} =\frac{1}{2}\left(\frac{3}{2}\right)^{n / 2} \rightarrow \infty \text{ cuando } n \rightarrow \infty.$$
  Para $n$ impar tenemos 
  $$\left|\frac{a_{n+1}}{a_n}\right|=\left|\frac{2^{\frac{n+1}{2}}}{3^{\frac{n+1}{2}}}\right|=\left(\frac{2}{3}\right)^{\frac{n+1}{2}} \rightarrow 0 \text{ cuando } n \rightarrow \infty.$$
  Entonces $\limsup_ {n \rightarrow \infty}\left|\frac{a_{n+1}}{a_n}\right|=\infty$, mientras que $\liminf_ {n \rightarrow \infty}\left|\frac{a_{n+1}}{a_n}\right|=0$. Nos encontramos así en el caso \textit{\textbf{c)}} del criterio del cociente (Teorema \ref{crit_cociente}), y por tanto este criterio no nos da información sobre la convergencia de $\sum a_n$.\\
  Ahora usemos el criterio de la raíz. Para $n$ par tenemos
  $$\sqrt[n]{\left|a_n\right|}=\left(\frac{1}{3^{\frac{n}{2}}}\right)^\frac{1}{n}=\frac{1}{3^\frac{1}{2}}=\frac{\sqrt{3}}{3}.$$
  Para $n$ impar tenemos
  $$\sqrt[n]{\left|a_n\right|}=\left(\frac{1}{2^\frac{n+1}{2}}\right)^\frac{1}{n}=\left(\frac{1}{2^\frac{n}{2}\cdot 2^\frac{1}{2}}\right)^\frac{1}{n}=\left(\frac{1}{2^\frac{1}{2}}\right)\left(\frac{1}{2^\frac{1}{2n}}\right)\to \frac{1}{2^\frac{1}{2}}=\frac{\sqrt{2}}{2} \text{ cuando }n\to\infty.$$
  Por lo tanto $\limsup _{n \rightarrow \infty} \sqrt[n]{\left|a_n\right|}=\frac{\sqrt{2}}{2}<1$, y por el criterio de la raiz, la serie $\sum a_n$ converge absolutanente.
\end{example}
\begin{example}
  Consideremos la serie
  \begin{align*}
    \sum a_n &= \frac{1}{2}+1+\frac{1}{8}+\frac{1}{4}+\frac{1}{32}+\frac{1}{16}+\frac{1}{128}+\frac{1}{64}+\dots\\[2ex]
    &= \frac{1}{2}+\frac{1}{2^0}+\frac{1}{2^3}+\frac{1}{2^2}+\frac{1}{2^5}+\frac{1}{2^4}+\frac{1}{2^7}+\frac{1}{2^6}+\dots
  \end{align*}
  Para $n$ par tenemos 
  $$\left|\frac{a_{n+1}}{a_n}\right|=\left|\frac{2^{n-2}}{2^{n+1}}\right|=\frac{1}{4}\cdot \frac{1}{2}\cdot\frac{2^n}{2^n}=\frac{1}{8}.$$
  Para $n$ impar tenemos 
  $$\left|\frac{a_{n+1}}{a_n}\right|=\left|\frac{2^n}{2^{n-1}}\right|=2.$$
  Entonces $\limsup_ {n \rightarrow \infty}\left|\frac{a_{n+1}}{a_n}\right|=2$, mientras que $\liminf_ {n \rightarrow \infty}\left|\frac{a_{n+1}}{a_n}\right|=\frac{1}{8}$. Nos encontramos así en el caso \textit{\textbf{c)}} del criterio del cociente (Teorema \ref{crit_cociente}), y por tanto este criterio no nos da información sobre la convergencia de $\sum a_n$.\\
  Ahora usemos el criterio de la raíz. Para $n$ par tenemos
  $$\sqrt[n]{\left|a_n\right|}=\left(\frac{1}{2^{n-2}}\right)^\frac{1}{n}=\left(\frac{4}{2^{n}}\right)^\frac{1}{n}=\frac{1}{2}\cdot 4^\frac{1}{n}\to \frac{1}{2} \text{ cuando }n\to\infty.$$
  Para $n$ impar tenemos
  $$\sqrt[n]{\left|a_n\right|}=\left(\frac{1}{2^{n}}\right)^\frac{1}{n}=\frac{1}{2}.$$
  Por lo tanto $\limsup _{n \rightarrow \infty} \sqrt[n]{\left|a_n\right|}=\frac{1}{2}<1$, y por el criterio de la raiz, la serie $\sum a_n$ converge absolutanente.
\end{example}
\begin{remark}
  Los anteriores dos ejemplos muestran que el criterio de la raíz puede darnos información acerca de la convergencia de una serie en situaciones en las que el criterio del cociente no es concluyente. Esto lo resumimos diciendo que el criterio de la raiz es ``más poderoso" que el criterio del cociente.
\end{remark}


  \section{Criterio de Dirichlet y criterio de Abel}
    \input{Lema_Sum_Parc}
    \begin{theorem}[Criterio de Dirichlet]
  Sea $\sum a_{n}$ una serie de números complejos, tal que su sucesión de sumas parciales está acotada, y sea $\left\{b_{n}\right\}$ una sucesión decreciente que converge a $0$ (esto implica que sus términos son no negativos). Entonces $\sum a_{n} b_{n}$ converge.
\end{theorem}
\begin{proof}
  Sea $\left\{A_{n}\right\}$ la sucesión de sumas parciales de la serie $\sum a_{n}$. Sea $n \in \mathbb{Z}^{+}$ cualquiera. Existe $M \geq 0$ tal que $\left|A_{n}\right| \leq M$, es decir $-M \leq A_{n} \leq M$ y por tanto $-M b_{n+1} \leq A_{n} b_{n+1} \leq M b_{n+1}$. Como $\lim _{n \rightarrow \infty}(-M) b_{n+1}=\lim _{n \rightarrow \infty} M b_{n+1}=0$, tenemos
  $$
    \lim _{n \rightarrow \infty} A_{n} b_{n+1}=0.
  $$
  Así, por el lema anterior, para probar que $\sum a_nb_n$ converge, nos basta probar la convergencia de la serie $\sum A_{n}\left(b_{n+1}-b_{n}\right)$. Como $\left\{b_{n}\right\}$ es una sucesión decreciente, tenemos
  $$
    \begin{aligned}
    \left|A_{n}\left(b_{n+1}-b_{n}\right)\right| & =\left|A_{n}\right|\left|b_{n+1}-b_{n}\right| \\
    & =\left|A_{n}\right|\left(b_{n}-b_{n+1}\right) \\
    & \leq M\left(b_{n}-b_{n+1}\right).
    \end{aligned}
  $$
  Notemos que $\sum\left(b_{n}-b_{n+1}\right)$ es una serie telescópica convergente, pues $\sum\left(b_{n}-b_{n+1}\right)=b_{1}-\lim _{n \rightarrow \infty} b_{n+1}=b_{1}$, con to cual, por el criterio de comparación se sigue que $\sum\left|A_{n}\left(b_{n+1}-b_{n}\right)\right|$ converge y por ende también lo hace $\sum A_{n}\left(b_{n+1}-b_{n}\right)$, lo cual completa la prueba.
\end{proof}


    \begin{theorem}[Criterio de Abel]
  Sea $\sum a_{n}$ una serie convergente y $\{b_n\}$ una sucesión monótona convergente. Entonces $\sum a_{n} b_{n}$ converge.
\end{theorem}
\begin{proof}
  Sea $\left\{A_{n}\right\}$ la sucesión de sumas parciales de la serie $\sum a_{n}$. Como $\sum a_{n}$ converge, se tiene que $\lim _{n \rightarrow \infty} A_{n}$ existe y que $\left\{A_{n}\right\}$ es una sucesión acotada. Además, $\left\{b_{n}\right\}$ convenge, y por tanto $\lim _{n \rightarrow \infty} A_{n} b_{n+1}$ existe. Así, por el Lema \ref{sum_parc}, para completar la demostración basta probar la convergencia de la serie $\sum A_{n}\left(b_{n+1}-b_{n}\right)$. Sea $n \in \mathbb{Z}^{+}$ cualquiera. Existe $M \geq 0$ tal que $\left|A_{n}\right| \leq M$, luego $\left|A_{n}\left(b_{n+1}-b_{n}\right)\right| \leq M\left|b_{n+1}-b_{n}\right|$. Si $\{ b_{n}\} \nearrow$ entonces 
  $$\sum\left|b_{n+1}-b_{n}\right|=\sum\left(b_{n+1}-b_{n}\right)=\lim _{n \rightarrow \infty} b_{n+1}-b_{1},$$
  y por tanto $\sum\left|b_{n+1}-b_{n}\right|$ converge, pues $\left\{b_{n}\right\}$ converge; si $\{b_n\}\searrow$ entonces 
  $$\sum\left|b_{n+1}-b_{n}\right|=\sum\left(b_{n}-b_{n+1}\right)=b_{1}-\lim _{n \rightarrow \infty} b_{n+1}$$ 
  y nuevamente tenemos que $\sum|b_{n+1}-b_{n}|$ converge. En todo caso, $\sum | b_{n+1}-b_{n} |$ converge; del criterio de comparación se sigue que $\sum\left|A_{n}\left(b_{n+1}-b_{n}\right)\right|$ converge, y por tanto también lo hace $\sum A_{n}\left(b_{n+1}-b_{n}\right)$, lo cual completa la prueba.
\end{proof}


\end{document}
