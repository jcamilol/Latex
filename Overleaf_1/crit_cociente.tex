\begin{theorem}[Criterio del cociente]\label{crit_cociente}
  Sea $\sum a_n$ una serie de números complejos no nulos, y tomemos
  \begin{equation*}
    r=\liminf_{n\to\infty}\left|\frac{a_{n+1}}{a_n}\right|, \hspace{1cm}R=\limsup_{n\to\infty}\left|\frac{a_{n+1}}{a_n}\right|.
  \end{equation*}
  Entonces se tiene que:
  \begin{itemize}
    \item[\textbf{a)}] Si $R<1$, la serie $\sum a_n$ converge absolutamente.
    \item[\textbf{b)}] Si $r>1$, la serie $\sum a_n$ diverge.
    \item[\textbf{c)}] Si $r\leq 1\leq R$, el criterio no decide.
  \end{itemize}
\end{theorem}
\begin{proof}
  \begin{itemize}
    \item[\textbf{a)}] Supongamos que $R<1$. Entonces existe $x \in \mathbb{R}$ tal que $R<x<1$. Tomemos $\text{\textepsilon}:=x-R>0$. Por la definición de $R$, existe $N \in \mathbb{Z}^{+}$ tal que para cualquier $n \geq N$ se tiene $\left|\frac{a_{n+1}}{a_n}\right|<R+\text{\textepsilon}=R+(x-R)=x$. Sea $n \geq N$ cualquiera. Como
    $x=\frac{x^{n+1}}{x^n} \quad$ (sabemos que $x \neq 0$ pues $R \geq 0$, ya que $\left\{\left|\frac{a_{n+1}}{a_n}\right|\right\}$ es una sucesión de términos positivos), se tiene $\left|\frac{a_{n+1}}{a_n}\right|<\frac{x^{n+1}}{x^n}$, y,
      $$
        \frac{\left|a_{n+1}\right|}{x^{n+1}}<\frac{\left|a_n\right|}{x^n} \leq \frac{\left|a_n\right|}{x^n} \text {, }
      $$
    con lo cual, para cualquier $n \geq N$ se tiene $\left|a_n\right| \leq c x^n$, donde $\quad c=\frac{\left|a_n\right|}{x^N} \in \mathbb{R}^{+}$. Tenemos que $\sum x^n$ es una sevie geométrica de radio $x \in(0, L)$, y por tanto $\sum x^n$ convenge. Así, por el criterio de companación se deduce que $\sum |a_n|$ converge, es decir, $\sum a_n$ converge absolutamente.
    \item[\textbf{b)}] Suponganos que $r>1$. Existe $x \in \mathbb{R}$ tal que $r>x>1$, Tomenos $\text{\textepsilon}=r-x>0$. Por la defintición de $r$, existe $N \in \mathbb{Z}^{+}$ tal que para todo $n \geq N$ se tiene $\left|\frac{a_{n+1}}{a_n}\right|>r-\text{\textepsilon}=r-(r-x)=x>1$, y por tanto $\left|a_{n+1}\right|>\left|a_n\right|$. Esto quiere decir que a partir de cierto punto, $\lbrace{|a_n|\rbrace}$ se comporta como una sucesión creciente de términos positivos, lo que implica $\lim _{n \rightarrow \infty}\left|a_n\right| \neq 0$ y por tanto $\lim _{n \rightarrow \infty} a_n \neq 0$, con lo cual $\sum a_n$ diverge.
    \item[\textbf{c)}] Para probar \textit{\textbf{c)}} consideremos los siguientes dos ejemplos: para la serie $\sum \frac{1}{n}$ se tiene
      $$
        \lim _{{n\rightarrow \infty}}\left|\frac{a_{n+1}}{a_n}\right|=\lim _{n \rightarrow \infty}\left|\frac{\frac{1}{n+1}}{\frac{1}{n}}\right|=\lim _{n \rightarrow \infty}\left(\frac{n}{n+1}\right)=1,
      $$
      luego $r=R=1$, y se tiene que $\sum \frac{1}{n}$ diverge; para la serie $\sum \frac{1}{n^2}$ se tiene
      $$
        \lim _{n \rightarrow \infty}\left|\frac{a_{n+1}}{a_n}\right|=\lim _{n \rightarrow \infty}\left|\frac{\frac{1}{(n+1)^2}}{\frac{1}{n^2}}\right|=\lim _{n \rightarrow \infty} \frac{n^2}{(n+1)^2}=1,
      $$
      y $r=R=1$, pero en este caso, por el Ejemplo \ref{Ejemplo_pseries}, se tiene que $\sum \frac{1}{n^2}$ converge.
  \end{itemize}
\end{proof}


